\documentclass{article}
\usepackage{times}
\usepackage[english,russian]{babel}
\usepackage[utf8]{inputenc}
\usepackage{dirtytalk}
\usepackage[a4paper, total={6in, 8in}]{geometry}
\usepackage{graphicx}
\usepackage{hyperref}


\graphicspath{ {./img/} }

\begin{document}
\title{Лабораторная работа 1}

\date{\today}
\maketitle

\say{Сначала учите науку программирования и всю теорию. Далее выработаете свой программистский стиль. Затем забудьте все и просто программируйте.
— George Carrette}

\section{Введение}

Цель: изучить работу Ethereum, изучить Ganache


\section{Основные понятия}

\begin{itemize}
	\item BlockChain (блокчейн) - выстроенная по определённым правилам непрерывная последовательная цепочка блоков (связный список), содержащих информацию. Чаще всего копии цепочек блоков хранятся на множестве разных компьютеров независимо друг от друга. 
	\item Etheruim (эфир) -  платформа для создания децентрализованных онлайн-сервисов на базе блокчейна
	\item Smart Contract (умный контракт) - компьютерный алгоритм, предназначенный для заключения и поддержания коммерческих контрактов в технологии блокчейн. 
	\item 	
\end{itemize}
























\begin{thebibliography}{9}

	\bibitem{lamport94}
	  \emph{Как работает Эфириум (Ethereum)?}
	  https://habr.com/post/407583/

\end{thebibliography}

\end{document}